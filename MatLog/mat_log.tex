% Этот шаблон документа разработан в 2014 году
% Данилом Фёдоровых (danil@fedorovykh.ru) 
% для использования в курсе 
% <<Документы и презентации в \LaTeX>>, записанном НИУ ВШЭ
% для Coursera.org: http://coursera.org/course/latex .
% Исходная версия шаблона --- 
% https://www.writelatex.com/coursera/latex/3.2

\documentclass[a4paper,12pt]{article}

%%% Работа с русским языком
\usepackage{cmap}					% поиск в PDF
\usepackage{mathtext} 				% русские буквы в формулах
\usepackage[T2A]{fontenc}			% кодировка
\usepackage[utf8]{inputenc}			% кодировка исходного текста
\usepackage[english,russian]{babel}	% локализация и переносы

%%% Дополнительная работа с математикой
\usepackage{amsmath,amsfonts,amssymb,amsthm,mathtools} % AMS
\usepackage{icomma} % "Умная" запятая: $0,2$ --- число, $0, 2$ --- перечисление

%% Номера формул
%\mathtoolsset{showonlyrefs=true} % Показывать номера только у тех формул, на которые есть \eqref{} в тексте.
%\usepackage{leqno} % Нумерация формул слева

%% Свои команды
\DeclareMathOperator{\sgn}{\mathop{sgn}}

%% Перенос знаков в формулах (по Львовскому)
\newcommand*{\hm}[1]{#1\nobreak\discretionary{}
	{\hbox{$\mathsurround=0pt #1$}}{}}

%%% Работа с картинками
\usepackage{graphicx}  % Для вставки рисунков
\graphicspath{{images/}{images2/}}  % папки с картинками
\setlength\fboxsep{3pt} % Отступ рамки \fbox{} от рисунка
\setlength\fboxrule{1pt} % Толщина линий рамки \fbox{}
\usepackage{wrapfig} % Обтекание рисунков текстом

%%% Работа с таблицами
\usepackage{array,tabularx,tabulary,booktabs} % Дополнительная работа с таблицами
\usepackage{longtable}  % Длинные таблицы
\usepackage{multirow} % Слияние строк в таблице

%%% Теоремы
\theoremstyle{definition} % Это стиль по умолчанию, его можно не переопределять.
\newtheorem{definition}{Определение}[section]
\newtheorem{ddefinition}{Определение}[subsection]
\newtheorem{dddefinition}{Определение}[subsubsection]
\newtheorem{remark}{Замечание}[section]
\newtheorem{rremark}{Замечание}[subsection]
\newtheorem{rrremark}{Замечание}[subsubsection]
\newtheorem{theorem}{Теорема}[section]
\newtheorem{ttheorem}{Теорема}[subsection]
\newtheorem{tttheorem}{Теорема}[subsubsection]
\newtheorem{proposition}{Предложение}[section]
\newtheorem{pproposition}{Предложение}[subsection]
\newtheorem{ppproposition}{Предложение}[subsubsection]
\newtheorem{lemma}{Лемма}[section]
\newtheorem{llemma}{Лемма}[subsection]
\newtheorem{lllemma}{Лемма}[subsubsection]
\newtheorem{upr}{Упражнение}[section]
\newtheorem{uupr}{Упражнение}[subsection]
\newtheorem{uuupr}{Упражнение}[subsubsection]
\newtheorem{example}{Пример}[section]
\newtheorem{eexample}{Пример}[subsection]
\newtheorem{eeexample}{Пример}[subsubsection]
\newtheorem{corollary}{Следствие}[section]
\newtheorem{ccorollary}{Следствие}[subsection]
\newtheorem{cccorollary}{Следствие}[subsubsection]
\newtheorem{axiom}{Аксиома}[section]
\newtheorem{aaxiom}{Аксиома}[subsection]
\newtheorem{aaaxiom}{Аксиома}[subsubsection]
\newtheorem{notation}{Обозначение}[subsection]

\theoremstyle{definition} % "Определение"
%\newtheorem{corollary}{Следствие}[theorem]
\newtheorem{problem}{Задача}[section]

\theoremstyle{remark} % "Примечание"
\newtheorem*{nonum}{Решение}

%%% Программирование
\usepackage{etoolbox} % логические операторы

%%% Страница
%\usepackage{extsizes} % Возможность сделать 14-й шрифт
\usepackage{geometry} % Простой способ задавать поля
\geometry{top=25mm}
\geometry{bottom=35mm}
\geometry{left=35mm}
\geometry{right=20mm}
%
\usepackage{fancyhdr} % Колонтитулы
\pagestyle{fancy}
\renewcommand{\headrulewidth}{0mm}  % Толщина линейки, отчеркивающей верхний колонтитул
\lfoot{Нижний левый}
\rfoot{Нижний правый}
\rhead{Верхний правый}
\chead{Верхний в центре}
\lhead{Верхний левый}
% \cfoot{Нижний в центре} % По умолчанию здесь номер страницы

\usepackage{setspace} % Интерлиньяж
%\onehalfspacing % Интерлиньяж 1.5
%\doublespacing % Интерлиньяж 2
%\singlespacing % Интерлиньяж 1

\usepackage{lastpage} % Узнать, сколько всего страниц в документе.

\usepackage{soulutf8} % Модификаторы начертания

\usepackage{hyperref}
\usepackage[usenames,dvipsnames,svgnames,table,rgb]{xcolor}
\hypersetup{				% Гиперссылки
	unicode=true,           % русские буквы в раздела PDF
	pdftitle={Заголовок},   % Заголовок
	pdfauthor={Автор},      % Автор
	pdfsubject={Тема},      % Тема
	pdfcreator={Создатель}, % Создатель
	pdfproducer={Производитель}, % Производитель
	pdfkeywords={keyword1} {key2} {key3}, % Ключевые слова
	colorlinks=true,       	% false: ссылки в рамках; true: цветные ссылки
	linkcolor=black,          % внутренние ссылки
	citecolor=green,        % на библиографию
	filecolor=magenta,      % на файлы
	urlcolor=blue           % на URL
}

%\renewcommand{\familydefault}{\sfdefault} % Начертание шрифта

\usepackage{multicol} % Несколько колонок


\newcommand{\comment}[1]{\left/\left/#1\right/\right/}
\newcommand{\real}{\mathbb{R}}
\newcommand{\N}{\mathbb{N}}
\newcommand{\Q}{\mathbb{Q}}
\newcommand{\I}{\mathbb{I}}
\newcommand{\Un}{\mathbb{U}}
\newcommand{\Chi}{\mathcal{X}}
\newcommand{\modul}[1]{\left|#1\right|}
\newcommand{\kvs}[1]{\left[#1\right]}
\newcommand{\krs}[1]{\left(#1\right)}
\renewcommand{\mod}{\ \mathrm{mod}\ }
\newcommand{\divv}{\,\mathrm{div}\,}
\renewcommand{\le}{\leqslant}
\renewcommand{\ge}{\geqslant}
\newcommand{\Z}{\mathbb{Z}}
\renewcommand{\d}{\partial}
\newcommand{\Ra}{\Rightarrow}
\newcommand{\La}{\Leftarrow}
\newcommand{\del}{\,\raisebox{-0.2ex}\vdots\,}
\newcommand{\LR}{\ \Leftrightarrow \ }
\newcommand{\LLR}{\ \Longleftrightarrow \ }
\newcommand{\fs}[1]{\left\{#1\right\}}
\DeclareMathOperator{\card}{card}
\newcommand{\trh}[2][0.8ex]{%
	#2%
	\vphantom{%
		\raisebox{#1}{\ensuremath{#2}}
		\raisebox{-#1}{\ensuremath{#2}}%
	}
}
\newcommand{\mystackrel}[1]{
	\stackrel
	{
		\begin{smallmatrix}
			#1
		\end{smallmatrix}
	}
	{\vphantom{\raisebox{0.5ex}{\ensuremath{\Longleftrightarrow}}}
		\Longleftrightarrow\vphantom{\raisebox{0.5ex}{\ensuremath{\Longleftrightarrow}}}}
}
\newcommand{\mystackreleq}[1]{
	\stackrel
	{
		\begin{smallmatrix}
			#1
		\end{smallmatrix}
	}
	{\vphantom{\raisebox{0.5ex}{\ensuremath{=}}}
		=\vphantom{\raisebox{0.5ex}{\ensuremath{=}}}}
}
\newenvironment{aggregate}{\left[\begin{matrix*}[l]}{\end{matrix*}\right.}

\newcommand{\wa}[1]{\href{#1}{\includegraphics[scale=0.2]{Wolfram_Alpha_logo.pdf}}}

\newcommand{\oR}{\overline{\real}}

\usepackage{enumerate}
\usepackage{datetime}
\usepackage{tikz}
\usetikzlibrary{calc}
\usetikzlibrary{arrows}
\usetikzlibrary{arrows.meta}
\usepackage{pgfplots}
\tikzset{>={stealth[width=2mm,length=3mm]}}
\usepackage{mathabx}
\usepackage{subfigure}
\usepackage{lscape}


%%% Заголовок
\author{
	\href{https://vk.com/victoriaisthebestgirl}{Фомина В.В.}\\ 
	Набрано в {\bf\LaTeX}
}
\title{
	Дебильник по предмету:\\
	 <<Математическая логика>>\\
	Четвертый семестр.
	\begin{center}
		\normalsize
		Специальность 02.03.03.\\
		Математическое обеспечение и администрирование информационных систем.\\
		Преподаватель - Григорьева Татьяна Матвеевна.\\
		Группа 244.\\
		Санкт-Петербург 2020.
	\end{center}
}
\date{Дата изменения: \today\quad\currenttime}

\pagestyle{empty}
\parindent = 0ex

\begin{document}
	
\maketitle
\newpage

\tableofcontents

\newpage

\section{Пропозициональные формулы. Таблицы истинности. Равносильные формулы. Основные равносильности. Тавтологии и противоречия.}

\section{Теоремы о представимости пропозициональной формулы с помощью формул, содержащих только три, две или одну логическую связку.}

\section{Теоремы о ДНФ и КНФ. Полином Жегалкина.}

\section{Понятия исчисления и формальной теории. Вывод, выводимая формула, полнота и непротиворечивость. Допустимое правило.}

\section{Секвенциальное исчисление высказываний. Допустимые правила секвенциального исчисления высказываний.}

\section{Теоремы о семантическом обосновании секвенциального исчисления высказываний.}

\section{Полнота и непротиворечивость секвенциального исчисления высказываний.}

\section{Метод резолюций для исчисления высказываний. Обоснование доказательства следования A1, ... , An $\Rightarrow$ B1, ... , Bk.}

\newpage

\section{Предикатные формулы: терм, атомарная формула, предикатная формула. Область действия квантора, свободные и связанные вхождения предметной переменной в формулу. Терм, свободный для подстановки в формулу вместо свободных вхождений предметной переменной.}

	\begin{definition} \ \\[1ex]
		\textbf{Предметная константа} - имя предмета.
	\end{definition}

	\begin{definition} \ \\[1ex]
		\textbf{Предметная переменная} - переменная, которая в качестве своих значений может принимать предметные константы.
	\end{definition}

	\begin{definition} \ \\[1ex]
		Символ $F$ в формальном языке является \textbf{функциональным символом}, если для любого символа $Х$, представляющий объект в языке, $F \krs{X}$ снова является символом, представляющим объект на этом языке
	\end{definition}

	\begin{definition}[\textbf{Терм}] \ \\
		\begin{enumerate}
			\item Предметная константа является термом.
			\item Предметная переменная является термом.
			\item Если $t_{1} , \ \dots , \ t_{n}$ -- термы, $f - n$-местный функциональный символ, то выражение $f \krs{t_{1} \ \dots , \ t_{n}}$ является термом.
			\item Никакие выражения, кроме полученных в результате применения п.п. 1 -- 3 этого определения, не являются термом.
		\end{enumerate}
	\end{definition}

	\begin{definition}[\textbf{Атомарная формула}] \
		\begin{enumerate}
			\item Если $t_{1} , \ \dots , \ t_{n}$ -- термы, $P - n$-местный предикатный символ, то $P \krs{t_{1}, \ \dots , \ t_{n}}$ является атомарной формулой.
			\item Никакие выражения, кроме полученных в результате применения п. 1 этого определения не являются атомарной формулой.
		\end{enumerate}
	\end{definition}

	\begin{definition}[\textbf{Предикатная формула}] \
		\begin{enumerate}
			\item Атомарная формула является предикатной формулой.
			\item Если А -- предикатная формула, то $\neg A$ является предикатной формулой.
			\item Если А, В -- предикатные формулы, * -- бинарная логическая связка, то $\krs{A * B}$ является предикатной формулой.
			\item Если А -- предикатная формула, $x$ -- предметная переменная, то $\forall x A$ и $\exists x A$ являются предикатными формулами.
			\item Никакие выражения, кроме полученных в результате применения п.п. 1 -- 4 этого определения, не являются предикатными формулами.
		\end{enumerate}
	\end{definition}
	
	\begin{definition} \ \\[1ex]
		\textbf{Кванторным комплексом} называется выражение вида $\forall x$ или $\exists x$, где $x$ -- имя предметной переменной.
	\end{definition}

	\begin{definition} \ \\[1ex]
		\textbf{Областью действия квантора} называется формула, стоящая непосредственно вслед за кванторным комплексом, содержащим это вхождение квантора.
	\end{definition}

	\begin{example} \ \\[1ex]
		$\forall x \underbrace{(P \krs{x, \ y, \ z} \ \rightarrow \ \exists y \underbrace{\forall z \ \underbrace{Q \krs{x, \ y, \ z}}_3}_2 )}_{1}$\\[1ex]
		Цифрами 1, 2, и 3 отмечены области действия соответсвенно квантора всеобщности по переменной $x$, квантора существования по переменное $y$ и квантора всеобщности по переменной $z$.
	\end{example}

	\begin{definition} \ \\[1ex]
		\textbf{Вхождение} предметной переменной в формулу называется \textbf{связанным}, если оно находится в кванторном комплексе или в области действия квантора по этой переменной.
	\end{definition}

	\begin{definition} \ \\[1ex]
		Вхождения предметных переменных, не являющиеся связанными, называются \textbf{свободными}.
	\end{definition}

	\begin{example} \ \\[1ex]
		Переменные, связанные одним и тем же квантором подчёркнуты одинаково.\\[1ex]
		$\forall \underline{x} (P(\underline{x}, \ y, \ z)) \rightarrow \ \exists \underline{\underline{y}} \forall \underline{\underline{\underline{z}}} Q(\underline{x}, \ \underline{\underline{y}}, \ \underline{\underline{\underline{z}}})$\\[1ex]
		Первые два вхождения предметных переменных $y$ и $z$ являются свободными.
	\end{example}

	\begin{definition}\ \\[1ex]
		Терм $t$ называется \textbf{свободным для подстанвки в формулу $\bf{F}$ вместо свободных вхождений предметной переменной $x$}, если $t$ не содержит переменных, в области действия кванторов по которым имеется свободное вхождение переменной $x$.
	\end{definition}

	\begin{definition}\ \\[1ex]
		Формула без свободных переменных называется \textbf{замкнутой}.
	\end{definition}

	\begin{definition}\ \\[1ex]
		Формула, у которой ни одна переменная не имеет как свободных, так и связанных вхождений, называется \textbf{чистой}.
	\end{definition}

\section{Интерпретации. Общезначимые и выполнимые формулы, противоречия.}
	Значение предикатной формулы можно вычислить, проинтерпретировав входящие в неё символы, т.е. задав содержательный смысл предметным константам, функциональным и предикатным символам.
	
	\begin{definition}\ \\[1ex]
		Для того, чтобы \textbf{задать интерпретацию формулы} достаточно
		\begin{itemize}
			\item задать область интерпретации $D$ -- множество констант;
			\item каждому $n$-местному функциональному символу $f$ поставить в соответсвие конкретную функцию из $D^{n}$ в $D$.
			\item каждому $n$-местному предикатному символу $P$ поставить в соотвествие конкретное отношение над $D^{n}$.
		\end{itemize}
	\end{definition}

	Значение атомарной формулы в заданной интерпретаци на заданном наборе значений входящих в неё свободных переменных вычисляется в соответствии с заданной интерпретацией.\\[1ex]
	Если вычислены значения формул $A$ и $B$ в заданной интерпретации на заданном наборе значений входящих в них свободных переменных, то значения формул $\neg A$ и $A * B$ вычисляются в соответсвие с таблицами истинности для логических связок $\neg$ и $*$.\\[1ex]
	Если $x, \ y_1, \ \dots , \ y_m$ -- список (быть может пустой) всех свободных переменных, входящих  в формулу $P(x, \ y_1, \ \dots , \ y_m)$, то для вычисления значения формулы $\forall x P(x, \ y_1, \ \dots , \ y_m) \ (\exists x P(x, \ y_1, \ \dots , \ y_m))$ на наборе значений $b_1, \ \dots , \ b_m$ свободных переменных $y_1, \ \dots , \ y_m$ достаточно вычислить значения $P(a, \ b_1 , \ \dots , \ b_m) \text{ при } a \in D$. Если при всех значениях $a \in D$ эти формулы истинны, то $\forall x P(x, \ b_1 , \ \dots , \ b_m)$ истинна, в противном случае она ложна (соответсвенно если при всех значениях $a \in D$ эти формулы ложны, то $\exists x P(x, \ b_1 , \ \dots , \ b_m)$ ложна, в противном случае она истнна).
	
	\begin{corollary}\ \\[1ex]
		Из способа вычисления значения предикатной формулы следует, что оно зависит только от значений свободных переменных.
	\end{corollary}

	\begin{definition}\ \\[1ex]
		Формула называется \textbf{истинной (ложной)} в заданной интерпретации, если она истинна (ложна) на всех наборах значений из области интерпретации, подставляемых вместо свободных вхождений переменных этой формулы.
	\end{definition}

	\begin{definition}
		Формула называется \textbf{выполнимой} в заданной интерпретации, если она истинна хоть на одном наборе значений из области интерпретации, подставляемых вместо свободных вхождений предметных переменных этой формулы.
	\end{definition}

	\begin{definition}\ \\[1ex]
		Формула называется \textbf{общезначимой (противоречием)}, если она истинна (ложна) в любой интерпретации.
	\end{definition}

	\begin{definition}\ \\[1ex]
		Формула называется \textbf{выполнимой}, если она выполнима хоть в одной интерпретации.
	\end{definition}

	То есть формула выполнима, если хоть в одной интерпретации хоть на одном наборе значений свободных переменных из области интерпретации она истинна.
	
	\begin{definition}\ \\[1ex]
		Формула $B$ логически следует из формул $A_1, \ \dots , \ A_n$, если в любой интерпретации на любом наборе значений свободных переменных, для которых все формулы $A_1, \ \dots , \ A_n$ истинны, формула $B$ тоже истинна.
	\end{definition}

	\begin{notation}
		$A_1, \ \dots , \ A_n \Rightarrow B$
	\end{notation}
	При этом $(A_1, \ \dots , \ A_n \Rightarrow B) \Leftrightarrow (A_1 \ \& \dots \& \ A_n \rightarrow B$ общезначима).
	
	\begin{notation}\ \\[1ex]
		$[P]_{t}^{x}$ -- результат подстановки терма $t$ в формулу $P$ вместо всех свободных вхождений переменной $x$.
	\end{notation}

\section{Смысл формулы с n свободными переменными в заданной интерпретации.}
	\begin{definition}[Высказывание]\ \\[1ex]
		\textbf{Высказыванием} называется утверждение, относительно которого можно сказать истинно оно или ложно.
	\end{definition}

	Значение предикатной формулы зависит только от значений свободных переменных.
	
	\begin{definition}[Замкнутая формула]\ \\[1ex]
		Формула без свободных переменных называется \textbf{замкнутой}.
	\end{definition}
	
	\begin{itemize}
		\item замкнутая формула задаёт высказывание
		\item формула с одной свободной переменной задаёт свойство объектов из $D$
		\item формула с $n$ свободными переменными задат $n$-местное отношение между объектами из $D$
	\end{itemize}
	
\section{Секвенциальное исчисление предикатов. Необходимость соблюдения ограничений на кванторные правила (примеры).}

\section{Полнота и непротиворечивость секвенциального исчисления предикатов.}

\section{Метод резолюций для исчисления предикатов. Обоснование доказательства следствия A1, ... ,	An $\Rightarrow$ B1, ... , Bk.}

\section{Понятие формальной теории. Формальные теории с равенством (примеры). Аксиомы для	равенства и аксиомы согласования с равенством.}

\section{Формальная арифметика (аксиоматическая теория чисел).}

\section{Первая теорема Геделя.}

\section{Вторая теорема Геделя.}

\section{Консервативность расширения формальной арифметики бесконечно большими числами.}

\section{Парадокс Рассела в наивной теории множеств. Его отсутствие в аксиоматических теориях	множеств.}

\section{Теория типов Рассела.}

\section{Аксиоматическая теория множеств Цермело-Френкеля.}

\section{Ординальные числа.}

\section{Конструктивные объекты. Формулы Бэкуса.}

\section{Примеры математических понятий алгоритма.}

\end{document}